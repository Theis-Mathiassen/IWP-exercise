\section{exercises}
\subsection{R1}
There is no difference between hosts and end systems, they are synonyms.
Some host systems, tvs, gaming consoles computers, server?

\subsection{R19}
$A->R_1->R_2->R_3->B$
$R_1 = 500kbps, R_2 = 2Mbps, R_3 = 1 Mbps$.

a)\\
The bottleneck of the system is $R_1$ and therefore 500kbps.

b)\\
$File = 4000 kb$.\\
$Time = \frac{4000}{500} = 8 seconds$.

c)\\
$R_2 = 100kbps$\\
$Time = \frac{4000}{100} = 40 seconds$.


\subsection{R23}
Layers of the internet protocol stack: (Se bogen side 80).
\begin{enumerate}
    \item Application
    \item Transport layer (TCP: Segments/reasemblies packets, flow control ... UDP: No flow control, no congestion control).
    \item Network layer
    \item Link layer
    \item Physical
\end{enumerate}

\subsection{R24}
Encapsulation: Happens at each step of the internet protocol. Where one layer packeges the recieved message and attaches a header.
De-encapsulation: Removes the header and sends the message down the internet protocol.

\subsection{P25}
\begin{enumerate}[label=\alph*)]
    \item $R = 5 Mbps$\\
    $d_prop = \frac{20000}{2.5*10^8} = 0.08 seconds$\\
    $Bandwidth delay product = R * d_prop = 400000 bits$
    \item Se svar a)
    \item The Bandwidth delay product is the number of bits on the line at once.
    \item $20000000 / 400000 = 50 meter$ 1 bit's "width" is 50 meter.
    \item $\frac{L}{R\frac{L}{c}} = \frac{L}{R\frac{1}{c}L}=\frac{1}{R\frac{1}{c}}=\frac{1}{\frac{R}{c}}=\frac{c}{R}$
\end{enumerate}





